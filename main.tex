\documentclass{article}
\usepackage[utf8]{inputenc}
\usepackage[super]{natbib}
\bibliographystyle{unsrtnat}

\title{Effect of EUSOL dressings on Healing Rate in Wounds by Secondary Intention: A prospective cohort study}
\author{Naveed Aman Pasha} 
\date{March 1, 2019}

\begin{document}

\maketitle

\begin{abstract}
Wound dressings are an essential part of care of surgical wounds left open to
heal by secondary intention as wounds that are not regularly dressed are prone
to delayed healing and accumulation of slough. Several materials have been used
to enhance the growth of healthy tissue in surgical wounds by chemically
debriding wound slough. One such material is the Edinburgh University Solution
of Lime which is commonly used in Pakistan. However, the efficacy of EUSOL gauze
in accelerating wound healing has not been proven in literature as compared to
normal saline soaked gauze. Moreover, it is an expensive solution. There is a
need to objectively establish the true benefit of EUSOL soaked dressings in
comparison to normal saline soaked dressing in order to.
\end{abstract}

\section{Introduction}
According to Wilkins et al., debridement, irrigation and cleaning for the
basis of wound care. Debridement involves mechanical removal of dead tissue,
irrigation involves application of fluid streams to the wound under pressure and
cleaning involves application of fluid to the wound.\cite{Wilkins_2013} These
activities aim to reduce the amount of non-viable biological material on the
surface of the wound. Non-viable biological material (such as fibrinous debris
of dead skin) inhibits myofibroblasts (that grossly appear as granulation
tissue) from growing into the wound. Wound care activities therefore increase
the chances of wound healing.

To that end, history has seen the application of various materials to wounds.
These materials range from herbal applications like turmeric to chemicals such
as betadine and povidone-iodine. Normal saline and EUSOL are two materials
commonly used in Pakistan. Normal saline consists of 0.9\% sodium chloride (or
152 mEq of NaCl) dissolved in water. It is colorless and odourless solution.
It's osmolarity is close to that of blood making it an `isotonic' fluid.

Smith et al.\ describe the Edinburgh University Solution of Lime (EUSOL) to be a
highly diluted weak acid comprising 1.99 percent EUSOL against 98.01 percent
water. The exact constituents of the Edinburgh University Solution of Lime are
0.54 percent hypochlorous acid, 1.28 percent calcium biborate and 0.17 percent
calcium chloride. Therefore, it is essentially formed of the chemical reaction
of boric and chlorate salts. In this chemical composition, EUSOL is volatile and
decomposes into hypochlorous acid and sodium chloride. This process is
accelerated in the presence of light therefore EUSOL is stored in
photo-resistant containers.\cite{Smith_1915}

\subsection{Mode of Action of EUSOL}
Edmiston et al.\ found normal saline to have the following advantages: it
hydrates the wound bed. It also reduces the burden of biological material
thereby expediting wound healing process.\cite{Edmiston_2016}

The antiseptic properties of EUSOL were known as far back as 1990s whereby
experiments show its effectiveness against streptococcus pyogenes among other
pathogens. EUSOL has even been investigated as an intravenous agent in the
treatment of the bubonic plague.\cite{connor1916eusol}

Recently, EUSOL has also come to be known as a chemical debridement
agent.\cite{Farrow_1991} It loosens slough (fibrinous, necrotic cellular debris)
from the base of the wound. In doing so, it prevents the retardation of wound
granulation (in-growth of new fibroblasts) and epithelialization (regeneration
of the outermost layer of the skin). This theoretically reduces the time to
wound healing. It also decreases the chance of wound infection as decreased
slough means reduced nidus of bacterial colonization.

\subsection{Mode of Action of Normal Saline}
Moisture is important for wound healing. Dry wounds have a greater tendency to
produce exudate (dead cellular debris) and thicker scabs (fibrous tissue). These
changes in turn delay epithelialization and prolong wound
healing.\cite{Winter_1963}

Application of normal saline to wounds (commonly known as `wet-to-dry
dressings`) keeps wounds moist thereby preventing the changes that prolong wound
healing. Normal saline is a physiological solution because it is isotonic and
this also prevents cellular damage. Over the passage of time the surface of the
wound desiccates making the normal saline increasingly hypertonic. This draws
water from the wound by osmosis ensuring that the wound stays
moist.\cite{Lim_2000}

\section{Rationale}
Wounds healing by secondary intention are at risk of developing a layer of
fibrinous exudate (commonly termed as `slough`). Slough has the potential to
arrest the formation of healthy granulation tissue as well as to elevate the
risk of wound super-infection. Meticulous wound dressing involves addressing the appearance of slough on wound in order to mitigate the risks of delayed wound healing and wound super-infection. The Edinburgh University Solution of Lime is commonly added to dressing routines as a method of chemical debridement of slough.

Several problems remain with the use of EUSOL.\@The foremost problem is that EUSOL is expensive with the cost of averaging to 3 rupees per milliliter. This price may not be problematic for small wounds that require once daily dressing. However, the price can climb steeply if the wounds are of large size or require multiple dressings within 24 hours.

Another problem, as detailed above, is that the chemical composition of EUSOL
degrades with time. Therefore, EUSOL cannot be purchased or stored in bulk
quantity. Not all patients have the capacity to visit a pharmacy repeatedly to
obtained more EUSOL when expired and this is especially a problem for patients living in remote areas.

Perhaps the most concerning point on using EUSOL for wound care is that its
efficacy has remained controversial for well more than a decade. One of the
first objections to the use of EUSOL for wound dressings appears in the
1990s.\cite{Burton_1992,Patton_1992} These objections are narrated in letters to
the editors where doctors describe events where they have been approached by
other healthcare providers either discouraging or actively refusing the use of
EUSOL in patient care. The NICE guidelines (NICE, 2019) categorically prohibits the use of EUSOL on wounds that are healing by secondary intention.

On the other hand, normal saline is significantly less costly and has no
expiration date. Therefore it is easier for patients to acquire and store normal
saline. Furthermore, normal saline dressings may have a debridement action as
well. When the normal saline dressing desiccates and the dry gauze is removed it peels the superficial layer of slough along with it.

However, it is difficult to synthesize an evidence based opinion regarding EUSOL from the literature. To the knowledge of the authors, there exists no study directly comparing EUSOL dressing with simple gauze soaked in normal saline. Rather, all available studies focus on comparison of EUSOL dressing with a variety of other dressing materials. Table 1 summarizes these studies (see Appendix 1).

It is immediately obvious that no unified conclusion can be drawn from these
studies. There is no single, uniform measure of wound healing across studies to
allow objective comparison of the relative performance of  EUSOL. Furthermore, none of the dressings materials described in these studies are commonly used in Pakistan for dressing wounds healing by secondary intention.

In summary, the rationale of this study is the need for effective and
inexpensive dressing to address the slough that frequently appears on surgical
wounds healing by secondary intention.

\subsection{Study Hypothesis}
Null: There is no difference in the healing rate of open surgical wounds
dressed with EUSOL dressings or with Normal Saline dressings by
secondary intention

Alternate: The healing rate of open surgical wounds dressed with EUSOL
dressings is greater than that of open surgical wounds dressed with
Normal Saline dressings by secondary intention

\subsection{Primary Objective}
To determine effectiveness of EUSOL dressing on the healing rate of open
surgical wounds compared to normal saline dressing by secondary
intention.

\section{Methods and Materials}
\subsection{Operational Definitions}
\begin{itemize}
\item \textbf{Wound Healing Rate}: The value $\theta$ of the delayed exponential curve
plotted on a graph of advance of wound margin towards the wound center against
time since surgery for a set of seven or more wound measurements such that:
\begin{enumerate}
\item The two longest, mutually perpendicular diameters of the wound, $a$ and $b$
measured in $mm$, are used to calculate the area of the wound, $S$ in $mm^2$,
using the formula

\[\frac{\pi}{4}a_i\cdot b_i\]

and the perimeter, $p$ in $mm$, of the wound using the formula

\[\pi[\frac{3}{4}(a_i+b_i)-\frac{1}{2}\sqrt{a_0\cdot b_0}]\]

\item The advance of wound margin, in mm, towards the center of the wound is
calculated using the formula

\[2\frac{S_0}{p_0T} mm/day\]

\item The seven measurements are taken at an interval of one week.\cite{Cukjati_2001}

\item The predicted time, in days, for a given wound to reduce to 5\% of its
initial area or the predicted time for a given wound to reduce to less than
100 $mm^2$ which ever is smaller. This definition has been adapted from
Cukjati et al.\cite{Cukjati_2001}
\end{enumerate}

\item \textbf{Normal Saline Dressing}: The practice of applying povidone-iodine to wound
edges followed by washing wounds with at least 500 cc of normal saline before
applying gauze in a clean or sterile fashion
\item \textbf{EUSOL Dressing}: The practice of applying povidone-iodine to wound edges
followed by washing wounds with at least 500 cc of normal saline before
applying gauze soaked in EUSOL in a clean or sterile fashion
\item \textbf{Open Surgical Wound}: Surgical wound where skin has not been approximated by
staples or sutures
\item \textbf{Wound Care Practitioner}: Wound Nurses, surgeons and/or surgical residents
with at least one year of experience in dressing surgical wounds healing by
secondary intention
\item \textbf{Diabetic Patient}: Patients with reduced ability to auto-regulate serum
glucose levels as defined by guidelines of the National Institute of Health
and Care Excellence, United Kingdom.\cite{ICGT_2015}:
\begin{enumerate}
\item Fasting blood glucose level $>$ 125 mg/dL
\item Random blood glucose level $>$ 200 mg/dL
\item HbA1c $>$ 6.5 mg/dL
\item Taking oral hypoglycemic agents
\item Taking subcutaneous insulin injection
\end{enumerate}
\item \textbf{Tobacco Usage}: Tobacco usage as defined in the Global Adult Tobacco Survey:\cite{GATS_2010}
\begin{enumerate}
\item \textbf{Daily user}: An adult who uses tobacco based products (smoked or
smokeless) every day.
\item \textbf{Less than daily user}: An adult who uses tobacco based products (smoked or
smokeless) but not every day.
\item \textbf{Never used}: An adult who has never smoked or smokeless tobacco products
in his or her lifetime.
\end{enumerate}
\end{itemize}


\subsection{Study Design}
This will be a single center, unmatched prospective cohort study.

\subsection{Study Setting}
This study setting will be the Aga Khan University Hospital, Karachi, Pakistan
whereby patients will be followed during both in-hospital stay as well as
outpatients for a total of six weeks after surgery.

\subsection{Study Procedures}
This study will involve the use of normal saline and EUSOL in their assessment
of wound healing capabilities.

Patients who have undergone any abdominal or limb surgery that result in an open
wound will be selected based on inclusion and exclusion criteria (see Inclusion
Criteria and Exclusion Criteria). Selected patients will be offered enrollment
in the study (See Appendix: Informed Consent). The number of patients who are
excluded from the study or who refuse to participate will be noted.

The following information will be noted for all selected patients using the "New
Patient Registration" (see 9):

\begin{itemize}
\item the patient's age and gender
\item the surgery that the patient underwent and the date of that surgery
\item whether the patient is a diabetic or a smoker
\item the longest dimension of the wound and the second-longest dimension that is perpendicular to the first
\item the type of dressing prescribed by the patient's primary surgeon (i.e. normal saline or EUSOL dressings)
\end{itemize}

Following initial registration, patients will be followed once a week for a
total of 7 weeks. At each follow-up, the following details will be recorded
using the "Follow-Up Wound Assessment" Form (see Appendix: Proforma)

\begin{itemize}
\item the longest dimension of the wound and the second-longest dimension that is perpendicular to the first
\item whether the consultant surgeon of that patient has decided to close the wound surgically on that visit and the date of that decision
\item whether the consultant surgeon of that patient has decided to retake the patient to OR for a re-look debridement and the date of that decision.
\end{itemize}

All wound assessments will be done by a wound nurse with at least one year of
experience or by a consultant surgeon. At the end of the wound assessment period
the collected data will be analyzed as per the analysis plan outlined below (See
Data Analysis).

The number of patients lost to follow-up and the number of patients who did not
receive the allocated treatment during the course of the study will be noted.
These patients will be analyzed on an intention-to-treat basis.

\subsection{Intra- and Inter-observer Variability Control}
In order to control variability, observers will be trained on two critical
aspects of wound measurement:
\begin{enumerate}
\item Avoiding parallax errors while taking wound measurements
\item Marking the extents of the two longest, perpendicular diameters; this will ensure that the wound is measured along the same axes on subsequent follow ups
\end{enumerate}

\subsection{Inclusion Criteria}
This study will include adult, post-operative patients with surgical wounds of
the abdomen and limbs that have been left to heal by secondary intention.

\subsection{Exclusion Criteria}
Patients with the following types of wounds will be excluded from this study:

\begin{itemize}
\item Wounds resulting from and/or complicated by viscero-cutaneous fistula: These wounds involve an abnormal connection between the epithelium of the skin and   the epithelium of a hollow viscus that normally produces a bodily fluid. Wound   care of viscero-cutaneous fistulas involves maneuvers to abate the physical and chemical effects of the bodily fluid to the skin. Such maneuvers have little or no connection with EUSOL. Therefore, wounds related to viscero-cutaneous fistulas are beyond the scope of this study.
\item Wounds resulting from pre-existing dermatological pathology, for example (but not limited to) psoriasis: Wounds resulting from pre-existing dermatological pathology have a different natural history of healing as compared to wounds on otherwise normal skin. Management of such wounds typically involves medical regimens tailored to curtail the pathology causing the wound and wound healing is directly correlated to controlling that pathology. Therefore, these wounds are beyond the scope of this study. 
\item Wounds in patients on corticosteroid therapy
\item Wounds in patients undergoing re-look debridement of the wound being studied
\end{itemize}

\subsection{Outcome Measure}
The main outcome measure of this study will be the difference in mean healing
rate between the normal saline and EUSOL groups. Healing rate will be calculated
as defined by Cukjati et al.\cite{Cukjati_2001}

\subsection{Sampling Technique}
This study will recruit patients using consecutive sampling. Patients will be
identified for selection on a daily basis using the Main Operating Room and Day
care Surgery Operating Room case list. Patients will be approached for informed
consent 24 - 48 hours after surgery.

\subsection{Sample Size}
The sample size for this study had been calculated on the basis of the works of
Carneiro et al. and Ramarao et al. who studied the wound healing rates after
normal saline and EUSOL dressings respectively as $mm^2$ per day. 
Carneiro et al. reported a wound healing rate of 149 $mm^2$ per day while using normal saline
dressings and Ramarao et al. reported a wound healing rate of 292 $mm^2$ per day
while using EUSOL dressings.

Based on this data, as well as using a 10\% inflation to account for loss to
follow up, the calculated sample size is 52 patients (26 patients in the normal
saline group and 26 patients in the EUSOL group). The sample size was calculated
using NCSS PASS software.

\subsection{Study Duration}
In order to estimate the study duration, the authors submitted a request to the
Aga Khan University Health Information Systems (HIS) in search of the number of
cases that resulted in surgical wounds of the abdomen and limbs left to heal by
secondary intention. The number of case records bearing the following ICD-9-CM
cods in the last year were queried from the hospital database via The Aga Khan
University Health Information Systems (HIS):

\begin{itemize}
\item Exploratory Laparotomy: 54.11 = 231
\item Re-open Laparotomy: 54.12 = 23
\item Reversal of Colostomy: 46.52 = 35
\item Carbuncle: 68.00 - 68.09 = 50
\item Incision and Drainage of Abscess: 86.04 = 176
\item Above Knee Amputation: 84.17 = 29
\item Below Knee Amputation: 84.15 = 62
\end{itemize}

This query resulted in a total of 606 cases which exceeds the required sample
size within one year. Accounting for the 7 week follow-up required for each
patent, this study is expected to take a maximum of 18 months to complete.

\subsection{Data Analysis}
By measuring wounds with the method as defined by Cukjati et al. (see
Operational Definitions), each wound will have a set of 7 measurements. These
measurements will then be plotted on a graph of progress of wound margins
towards the center of the wound against time. This graph follows a delayed
exponential curve having coefficient $\theta$. The coefficient $\theta$ will be
calculated for all patients by fitting their individual wound measurements to
the delayed exponential curve. Wounds having a faster healing rate will have a
higher value of $\theta$.

Student's t-test will be used to detect any significant difference in $\theta$ of
patients in normal saline group and those in the EUSOL group.

\bibliography{references}

\end{document}